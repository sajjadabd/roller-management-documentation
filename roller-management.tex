\documentclass[12pt]{article}

\usepackage{tabularx}
\usepackage[table]{xcolor}
\usepackage{multirow}

\newcolumntype{C}{ >{\centering\arraybackslash} m{5cm} }
\newcolumntype{E}{ >{\centering\arraybackslash} m{7cm} }
\newcolumntype{D}{ >{\centering\arraybackslash} m{3cm} }
\newcolumntype{F}{ >{\centering\arraybackslash} m{1cm} }
\newcolumntype{G}{ >{\centering\arraybackslash} m{2cm} }

\usepackage[margin=1.1in,footskip=.25in]{geometry}


\usepackage{tikz}
\usetikzlibrary{graphs,quotes,arrows.meta}
\usetikzlibrary{automata,positioning}
\usetikzlibrary{shapes,arrows}
\usetikzlibrary{chains}
\usetikzlibrary{matrix,backgrounds}
\usetikzlibrary{calc}




\usepackage{amsmath, amssymb}
\usepackage{mathtools}

\usepackage[most]{tcolorbox}

\tcbset{
    %frame code={}
    %center title,
    colback=gray!5!white,
    colframe=gray!75!black,
    toptitle=2mm,
    righttitle=2mm,
    bottomtitle=2mm,
    fonttitle= \bfseries\large,
    left=10pt,
    right=10pt,
    top=10pt,
    bottom=10pt,
    %colback=gray!5,
    %colframe=gray,
    width=\dimexpr\textwidth\relax,
    %enlarge left by=0mm,
    boxsep=5pt,
    %arc=0pt,outer arc=0pt,
}

\usepackage{xepersian}
\settextfont[Scale=1]{Shabnam}

\setlength{\parindent}{0pt}

\renewcommand{\baselinestretch}{1.5} 

\begin{document}


\section{غلطک}

- کد غلطک 

- جنس غلطک

- سختی غلطک

- مدل غلطک

- قطر قدیم غلطک 
\lr{(mm)}

- قطر جدید غلطک 
\lr{(mm)}

- عرض کالیبر 
\lr{(mm)}




\begin{align*}
\text{وضعیت غلطک }
\Rightarrow
\begin{cases}
\text{انبار workshop}
\Rightarrow
\begin{cases}
\text{خام}
\\
\text{تراش خورده}
\end{cases}
\\
\text{در حال تراش}
\\
\text{آماده به کار}
\Rightarrow
\begin{cases}
\text{مونتاژ}
\\
\text{سوار استند}
\end{cases}
\\
\text{در خط تولید}
\end{cases}
\end{align*}





\begin{align*}
\text{جایگاه غلطک }
\Rightarrow
\begin{cases}
\text{Roughing}
\\
\text{Intermediate}
\\
\text{Finishing}
\end{cases}
\end{align*}




- میزان مصرفی قطر غلطک نسبت به قطر اصلی بر حسب 
\lr{(mm)}




- میزان مصرفی قطر غلطک نسبت به قطر اصلی بر حسب درصد


- میزان مصرفی قطر غلطک نسبت به قطر قبلی بر حسب درصد



- وضعیت غلطک از نظر قطر


- وضعیت غلطک از نظر سختی





\begin{align*}
\text{ تناژ کالیبر }
\Rightarrow
\begin{cases}
\text{کالیبر 1}
\\
\text{کالیبر 2}
\\
\text{کالیبر 3}
\\
\text{کالیبر 4}
\end{cases}
\end{align*}




- تناژ استاندارد کالیبر




\begin{align*}
\text{ وضعیت کالیبر }
\Rightarrow
\begin{cases}
\text{در حال استفاده}
\\
\text{آماده به کار}
\\
\text{غیر قابل استفاده}
\end{cases}
\end{align*}




- مقدار کیلوگرم بر تناژ غلطک


- مقدار جرم تراش شده ی قبلی


- مقدار جرم تراش شده ی کل





\newpage

\section{استند}



-غلطک بالا

-غلطک پایین

- کد استند

- وضعیت استند


- \lr{Center Distance (mm) : }
مرکز تا مرکز غلطک ، برای زاویه ی میل گاردان کاربرد دارد



- \lr{Air Gap (mm) : }
فاصله ی بین دو غلطک ، اندازه ی فلز محصول را تعیین می کند




\begin{align*}
\left.
\begin{array}{r}
\text{ 
تناژ کارکرد غلطک بالا
\lr{(ton)}
 }  
\\
\text{ 
تناژ کارکرد غلطک پایین
\lr{(ton)}
 }  
\\
\text{ 
تناژ کارکرد غلطک ورتیکال سمت آزاد
\lr{(ton)}
 }  
\\
\text{ 
تناژ کارکرد غلطک ورتیکال سمت موتور
\lr{(ton)}
 }  
\\
\end{array}
\right\}
\end{align*}






\begin{align*}
\left.
\begin{array}{r}
\text{ مقدار تراش غلطک بالا }  
\\
\text{  مقدار تراش غلطک پایین }
\\
\text{ مقدار تراش غلطک ورتیکال سمت آزاد }
\\
\text{ مقدار تراش غلطک ورتیکال سمت موتور }
\\
\end{array}
\right\}
\end{align*}






\begin{align*}
\left.
\begin{array}{r}
\text{ 
مقدار جرم تراش شده ی غلطک بالا
\lr{(kg)} 
}  
\\
\text{ 
مقدار جرم تراش شده ی غلطک پایین
\lr{(kg)}
 }  
\\
\text{ 
مقدار جرم تراش شده ی غلطک ورتیکال سمت آزاد
\lr{(kg)}
 }  
\\
\text{ 
مقدار جرم تراش شده ی غلطک ورتیکال سمت موتور
\lr{(kg)}
 }  
\\
\end{array}
\right\}
\end{align*}







\begin{align*}
\left.
\begin{array}{r}
\text{ مقدار کیلوگرم بر تناژ غلطک پایین }  
\\
\text{ مقدار کیلوگرم بر تناژ غلطک بالا }  
\\
\text{ مقدار کیلوگرم بر تناژ غلطک ورتیکال سمت آزاد }  
\\
\text{ مقدار کیلوگرم بر تناژ غلطک ورتیکال سمت موتور }  
\\
\end{array}
\right\}
\Leftarrow
\text{مهمترین فاکتور مدیریت غلطک}
\end{align*}









\section{ایجاد غلطک}




\begin{tikzpicture}[level distance=1.5cm,
  level 1/.style={sibling distance=9cm},
  level 2/.style={sibling distance=1.5cm}]
  \node {سکشن / بارمیل}
    child {node {
\text{
گرد
 میله 
 مختلف
های
سایز
}
}
      child {node {\lr{ Roughing / Intermediate / Finishing }}
		child { node {
\text{ 
جنس
، قطر
، کالیبر 
چنتا 
}
} }
}
    }
    child {node { 
\text{
 \lr{ 14/16/18/20 }
های
 سایز
 تیرآهن
، ناودونی 
، نبشی 
 }
}
      child {node { \lr{Reversible , Due , Uni} } 
		child { node {
\text{ 
جنس
، قطر
، کالیبر 
چنتا 
}
} }
}
    };
\end{tikzpicture}





\section{گزارش تعویض غلطک ها}


- کد غلطک

- تغییر وضعیت غلطک

- محل قرارگیری غلطک

- توضیحات

- تاریخ




\section{گزارش تعویض استند ها}



- کد استند

- وضعیت استند

- تاریخ 




\newpage

\section{گزارش تراش غلطک ها}



- کد غلطک

- تاریخ

- قطر قدیم غلطک

- قطر جدید غلطک

- میزان تراش کلی


\begin{align*}
\text{ میزان تراش کالیبرها }
\Rightarrow
\begin{cases}
\text{کالیبر 1}
\\
\text{کالیبر 2}
\\
\text{کالیبر 3}
\\
\text{کالیبر 4}
\end{cases}
\end{align*}







\section{نکات مهم}


** وقتی تمام کالیبر های یک غلطک تناژ خود را میزنند ، غلطک می رود برای تراشکاری


** غلطک تا میزان قطر مینیمم تراش می خورد و بعد از اینکه به مینیمم قطر خود رسید اسقاط می شود ، اینجا باید
 $ \cfrac{kg}{ton} $
  آن را به دست آورد ، بهترین غلطک های دنیا
$ 0.2 \frac{kg}{ton} $

مثلاً اگر 500000 تن در سال محصول می خواهیم ، 
\lr{x}
را ضربدر 
$0.2$
می کنیم تا به دست بیاوریم چند تا غلطک نیاز داریم













\end{document}