\documentclass[12pt]{article}

\usepackage{tabularx}
\usepackage[table]{xcolor}
\usepackage{multirow}

\newcolumntype{C}{ >{\centering\arraybackslash} m{5cm} }
\newcolumntype{E}{ >{\centering\arraybackslash} m{7cm} }
\newcolumntype{D}{ >{\centering\arraybackslash} m{3cm} }
\newcolumntype{F}{ >{\centering\arraybackslash} m{1cm} }
\newcolumntype{G}{ >{\centering\arraybackslash} m{2cm} }

\usepackage[margin=1.1in,footskip=.25in]{geometry}

\usepackage{amsmath, amssymb}
\usepackage{mathtools}

\usepackage[most]{tcolorbox}

\tcbset{
    %frame code={}
    %center title,
    colback=gray!5!white,
    colframe=gray!75!black,
    toptitle=2mm,
    righttitle=2mm,
    bottomtitle=2mm,
    fonttitle= \bfseries\large,
    left=10pt,
    right=10pt,
    top=10pt,
    bottom=10pt,
    %colback=gray!5,
    %colframe=gray,
    width=\dimexpr\textwidth\relax,
    %enlarge left by=0mm,
    boxsep=5pt,
    %arc=0pt,outer arc=0pt,
}

\usepackage{xepersian}
\settextfont[Scale=1]{Shabnam}

\setlength{\parindent}{0pt}

\renewcommand{\baselinestretch}{1.5} 

\begin{document}


\section{غلطک}

- کد غلطک 

- جنس غلطک

- سختی غلطک

- مدل غلطک

- قطر قدیم غلطک 
\lr{(mm)}

- قطر جدید غلطک 
\lr{(mm)}

- عرض کالیبر 
\lr{(mm)}




\begin{align*}
\text{وضعیت غلطک }
\Rightarrow
\begin{cases}
\text{انبار workshop}
\Rightarrow
\begin{cases}
\text{خام}
\\
\text{تراش خورده}
\end{cases}
\\
\text{در حال تراش}
\\
\text{آماده به کار}
\Rightarrow
\begin{cases}
\text{مونتاژ}
\\
\text{سوار استند}
\end{cases}
\\
\text{در خط تولید}
\end{cases}
\end{align*}





\begin{align*}
\text{جایگاه غلطک }
\Rightarrow
\begin{cases}
\text{Roughing}
\\
\text{Intermediate}
\\
\text{Finishing}
\end{cases}
\end{align*}




- میزان مصرفی قطر غلطک نسبت به قطر اصلی بر حسب 
\lr{(mm)}




- میزان مصرفی قطر غلطک نسبت به قطر اصلی بر حسب درصد


- میزان مصرفی قطر غلطک نسبت به قطر قبلی بر حسب درصد



- وضعیت غلطک از نظر قطر


- وضعیت غلطک از نظر سختی





\begin{align*}
\text{ تناژ کالیبر }
\Rightarrow
\begin{cases}
\text{کالیبر 1}
\\
\text{کالیبر 2}
\\
\text{کالیبر 3}
\\
\text{کالیبر 4}
\end{cases}
\end{align*}




- تناژ استاندارد کالیبر




\begin{align*}
\text{ وضعیت کالیبر }
\Rightarrow
\begin{cases}
\text{در حال استفاده}
\\
\text{آماده به کار}
\\
\text{غیر قابل استفاده}
\end{cases}
\end{align*}




- مقدار کیلوگرم بر تناژ غلطک


- مقدار جرم تراش شده ی قبلی


- مقدار جرم تراش شده ی کل









\end{document}